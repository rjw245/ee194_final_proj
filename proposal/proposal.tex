%%% Setup %%%
\documentclass{article} \usepackage{fullpage} \usepackage{url} \title{Proposal:
A Comparison of Concurrency in Rust and C} \author{ Josh Pfosi\\ \and Riley
Wood\\ \and Henry Zhou\\ } \begin{document} \maketitle

%%% The Meat %%%

\section*{Background}

Computer architects today are faced with a power wall which limits the trend of
rising clock frequency seen over the past two decades. Yet the number of
transistors on chip continues to double every two years as predicted by Moore's
Law, architects are left with this question: How can we continue to improve
processor performance without scaling clock speed? The industry has answered
with multicore processors~\cite{Larus:2009}. The industry has begun to favor
parallelism over clock frequency. This shift has significant ramifications for
programmers. A program's performance scales trivially with clock frequency,
whereas parallel programs oftentimes require significant design effort on the
part of the programmer to scale well. As of now, multicore machines burden the
programmer with efficient usage of the parallel cores~\cite{Sutter:2005}. It is difficult for most
programmers to decompose an algorithm into parallel workloads and to anticipate
all of the interactions that can occur. This proliferates new kinds of errors
that are much harder to anticipate, find, and prevent. These include deadlock,
livelock, and data races. It is important that programmers start writing
parallel programs in order to scale software performance with hardware. To this
end, programmers should feel confident in their ability to write safe, parallel
code. The question becomes: Which tools currently available are best suited for
this task?

Rust is a new systems programming language that facilitates writing safe
concurrent programs through an advanced type system~\cite{rust-lang.org}. It
introduces the idea of data ownership where variable bindings ``own'' the data to
which they point. This and other statically enforced safety features reduce the
burden of concurrent programming on the programmer. Contrast this with other
standard concurrent programming models such as OpenMP, \texttt{pthreads}, and C++11's
\texttt{std::threads}, which are difficult to learn and error-prone, Rust's
approach is novel.

To investigate the research question, we have selected Rust and C's
\texttt{pthreads} as representatives for contemporary approaches to parallelism.
Namely, Rust, which relies on static analysis, and \texttt{pthreads}, which relies on
programmer expertise, project-specific conventions, and manual synchronization.
The specific contribution made by this research will be to glean whether Rust's
strict compile-time rules and ``extra runtime book-keeping''~\cite{rust-lang.org} have an adverse
effect on performance.

\section*{Goal and Description}

Rust's concurrency constructs and paradigms make writing concurrent code safer
and easier by preventing code with common concurrency errors from compiling. We
are interested to see if Rust's new constructs affect its performance in a
significantly detrimental way when compared with C, a widely-used
concurrency-capable language. Metrics we are interested in include program
execution time, power consumption, and idle time. We will present our findings
through a presentation and written report.

Our findings should be useful to systems programmers who are evaluating languages
for their next project. With our results, they should be able to make a more
educated decision about which systems language, C or Rust, provides a better
trade-off between programmer productivity and performance.

%Add why our findings will be useful to others

\section*{Methodology}

Rust and C differ in many ways, but we seek to isolate the effect of their
concurrency models on performance. We will do this in the following way:

We will choose a set of benchmarks that can be run both serially and
concurrently and implement them in C and Rust. We will compare the performance
of the two languages on the single-threaded benchmarks to establish a baseline
for how Rust and C differ. Then any further differences that arise in our
multi-threaded tests can roughly be attributed to each language's concurrency
model. Our benchmarks will be run in a multithreaded-capable simulator. We will
vary parameters such as input size and the number of threads and collect metrics
such as total execution time, CPU idle time, and power consumption.  

Furthermore, we plan on exploring the differences in performance observed using
different hardware. For this section, we will execute concurrent benchmarks at
different thread counts with different hardware settings such as L1/L2 cache
size and cache associativity. We seek to examine if changes in relative
performance between C and Rust arise when running benchmarks on different
hardware.


\section*{Equipment and Tools}

We will be writing code in C and Rust. For more convincing analysis, we will
compile our C code using both clang and gcc. Rust will be compiled with the
standard rustc compiler. We are still in the process of selecting a suite of
concurrent benchmarks to use. Once we have chosen a set of benchmarks and ported
them to Rust, we will run our simulations in the Sniper Multi-Core Simulator
\cite{Sniper} on the Linux compute servers hosted at Tufts University.

%benchmark ideas: matrix multiplication Josh's Senior Design benchmark parsec 

\section*{Timeline}

\begin{table} [h!]
  \centering
  \begin{tabular} {|l|l|}
    \hline
    April 11 & Select benchmarks.                                                \\ \hline
    April 15 & Implement benchmarks in C and Rust and setup testing environment. \\ \hline
    April 20 & Project status discussion with Dr. Hempstead                      \\ \hline
    April 22 & Completed all benchmarks and collect all data                     \\ \hline
    April 27 & Project Oral Presentation                                         \\ \hline
    May 3    & Final Report Submission                                           \\ \hline
  \end{tabular}
\end{table}


%%% References %%%
\nocite{*}
\bibliographystyle{IEEEtran}
\bibliography{IEEEabrv,proposal}

%%% The End %%%
\end{document}
