%%% Setup %%%
\documentclass{article}
\usepackage{fullpage}
\usepackage{url}
\title{Proposal: A Comparison of Concurrency in Rust and C}
\author{
    Josh Pfosi\\
    \and
    Riley Wood\\
    \and
    Henry Zhou\\
}
\begin{document}
\maketitle

%%% The Meat %%%

\section*{Background}

Computer architects today are faced with a power wall. CPU clock speeds cannot increase any further, or else they will be unable to be cooled. Yet the number of transistors onchip continues to double every two years as predicted by Moore's Law, so the question to architects is this: what can we spend additional transistors on, if not clock speed? The industry has answered this question by converging on multicore computers: computers that integrate multiple processor cores into
one chip \cite{Larus:2009}. The industry is forced to favor increased parallelism over increased speed.

This shift has huge ramifications for programmers. A program's performance improves automatically as clock speed increases, but programs need to be written specially to take advantage of hardware parallelism. As of now, multicore machines put the burden of utilizing parallel cores on the programmer, and writing parallel programs is difficult. It is hard for most people to decompose a program into parallel workloads and think about all of the interactions that can occur as they
are executing at once. This makes the new types of errors that arise with concurrency that much harder to anticipate, find, and prevent. These include deadlock, livelock, and pointer aliasing, to name a few. It's important that programmers start writing parallel programs in order to take advantage of all that Moore's Law offers, but for this to happen, programmers need to feel confident in their ability to write safe, parallel code.

Highlight how writing concurrent programs is still difficult, prone to error. Rust is a new systems programming language that attempts to make writing concurrent programs easier for programmers through its type system. 

Rust is a new systems programming language which guarantees memory safety through a complex type system. 

Discuss Rust's concurrency constructs. Rust uses certain types to perform "extra runtime bookkeeping" on shared values, which we believe could negatively impact performance \cite{rust-lang.org}. Discuss pthreads in C.


\section*{Goal and Description}

Rust's new concurrency constructs make writing concurrent code safer and easier by preventing concurrency errors at compile time. We are interested in comparing the performance of threading in Rust and in C to see if Rust's new constructs affect its peformance in anyway. In other words, Rust helps programmers write concurrent code, but at what cost to performance?

What metrics are we interested in?


\section*{Methodology}

Rust and C are inherently different in many ways, so we will need to isolate how each one's implementations of concurrency impact performance. We will do this in the following way:

We will choose a set of benchmarks that can be run both serially and parallelized and implement them in C and Rust. We will use the data colllected from the single-threaded benchmarks to establish a baseline for how the two languages differ. Then, any further differences in performance that arise during the multithreaded tests can be attributed to each language's implementation of concurrency.


\section*{Equipment and Tools}

We will be writing code in C and Rust. For more convincing analysis, we will compile our C code using both clang and gcc. Rust will be compiled with the standard rustc compiler. We are still in the process of selecting a suite of benchmarks to use. Once we have chosen a set of benchmarks and ported them to Rust, we will run our simulations in the SNIPER Multi-Core Simulator \cite{Sniper} on the Linux compute servers hosted at Tufts University.


\section*{Timeline}


%%% References %%%
\bibliographystyle{unsrt}
\bibliography{proposal}

%%% The End %%%
\end{document}
